\section{Introduction}\label{sec:introduction}

A wireless sensor network (\textsc{wsn}) consists of collaborating embedded devices (sensor nodes) with capabilities of sensing, computation and communication
\cite{sudevalyam2010energy}. Wireless sensor networks can deployed for a plethora of purposes such as habitat \cite{mainwaring2002wireless},
earthquake \cite{suzuki2007earthquake} and health \cite{saadaoui2007architecture} monitoring.

Most sensor nodes do not depend on existing infrastructure and are powered by batteries, which makes initial deployment in remote locations
convenient. On the other hand, the lifetime of these embedded devices is limited by the amount of energy that can be stored in the batteries. Furthermore, 
the number of sensors and their locations might render the activity of replacing nodes' batteries unfeasible \cite{moser2007real}. 

Intensive research has been conducted on energy harvesting as a way to extend the lifetime of wireless sensor networks. Several types of energy such as
solar, eolic (wind), vibrational, and thermal among others can be scavenged from the surroundings of a sensor node to replenish its battery \cite{roundy2004power}. 
Promising as it may seem, energy harvesting poses new challenges to the scientific community \cite{lu2010accurate}:

\begin{itemize}
	\item Environmental energy sources behave stochastically, which means it is not possible to know beforehand the exact amount of available energy.
	\item Conventional task scheduling techniques were not designed for energy-limited scenarios and cannot deal properly with the uncertainty in energy availability.
\end{itemize}

This paper evaluates the performance of two task scheduling techniques, Earliest Deadline First (\textsc{edf}) and the Lazy Scheduling
Algorithm (\textsc{lsa}), for a single node under energy-limited conditions. The two techniques were conceived for unlimited and limited energy scenarios respectively \cite{moser2007real}. We also present a new scheduling technique, the Smooth to Averate Method
(\textsc{stam}), designed to handle bursts in power consumption and address both time and energy constraints adequately.  

We consider solar energy scavenging through photo-voltaic conversion, as it provides the highest power density of conventional environmental energy harvesting techniques
\cite{raghunathan2005design}. We use a Markov model based on a matrix of transition probabilities for three radiation intensity states to predict
the power provided by the harvesting unit \cite{poggi2000stochastic}.