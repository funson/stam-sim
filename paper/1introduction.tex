\section{Introduction}\label{sec:introduction}

A wireless sensor network (\textsc{wsn}) consists of collaborating sensor nodes with capabilities of sensing, computation and communication~\cite{sudevalyam2010energy}. Wireless sensor networks can be deployed for a plethora of purposes such as habitat monitoring~\cite{mainwaring2002wireless}, earthquake detection~\cite{suzuki2007earthquake}, and healthcare~\cite{saadaoui2007architecture}. 

To make deployment easy, wireless sensor networks usually do not rely on existing infrastructure, and sensor nodes are powered by batteries. On the other hand, the lifetime of these embedded devices is limited by the amount of energy that can be stored in the batteries. Furthermore, the number of sensors and their locations might render the activity of replacing nodes' batteries unfeasible or very costly~\cite{moser2007real}. 

To solve the above problem, intensive research has been conducted on energy harvesting as a way to extend the lifetime of wireless sensor networks. Several types of energy such as solar, eolic (wind), vibrational, and thermal among others can be scavenged from the surroundings of a sensor node to replenish its battery~\cite{roundy2004power}. Promising as it may seem, energy harvesting poses new challenges to the scientific community~\cite{lu2010accurate}:

\begin{itemize}
	\item Environmental energy sources behave stochastically, making the accurate predication on incoming energy amount very difficult.
	\item Conventional task scheduling techniques were not designed for energy-limited scenarios and cannot deal properly with the uncertainty in energy availability.
\end{itemize}

It has been pointed out that the traditional scheduling method, Earliest Deadline First (\textsc{edf}), may not work well under energy-limited conditions~\cite{moser2007real}, and as such new algorithms such as the Lazy Scheduling Algorithm (\textsc{lsa}) has been proposed to ``solve" the problem~\cite{moser2007real}. Although it has been theoretically proved that LSA is optimal, it requires a good predication on the incoming energy source to operate well. We find that energy predication is non-trivial, and it is nearly impossible to implement any ``intelligent" learning algorithm for such purpose over tiny sensors due to the limited computational resource. Installing a pre-trained energy predication model does not work either, because such a model depends on where and when the model was built and may not work well when the sensors are deployed in different places and function over a long time period. 

In this paper, we make contributions by proposing two new scheduling techniques, the Smooth to Average Method (\textsc{stam}) and Smooth to Full Utilization (\textsc{stfu}), to handle the energy uncertainty and deadline constraint without relying on any energy predication model\footnote{Although in the later of the paper, we build an energy charging model for solar energy harvesting. This model is \textit{purely} for the purpose of performance comparison and in real implementation such a model is not required.}.
We consider solar energy scavenging through photo-voltaic conversion, as it provides the highest power density of conventional environmental energy harvesting techniques~\cite{raghunathan2005design}. 

%We use a Markov model based on a matrix of transition probabilities for three radiation %intensity states to predict the power provided by the harvesting unit \cite
%{poggi2000stochastic}.