\section{Simulation Evaluation} \label{sec:simulation}

We have developed a simulation framework for comparing the \textsc{STAM} task scheduling to traditional scheduling algorithms.  Our simulation includes a stochastic energy harvesting process, a random task list and \textsc{STAM} task list generator, the scheduling processes, and an execution process.  We execute $n$ simulations on one task list per run, and generate task lists for $r$ runs.  Each task list consists of $k$ tasks.  The framework is as follows:

\begin{algorithm}[h]
\begin{algorithmic}
\FOR{$i = 1$ \TO $r$}
\STATE $setRandomSeed(i + offset)$
\STATE $taskList, stamTasks \gets generateTasks()$
\STATE $schedule \gets schedule(taskList)$
\STATE $stamSchedule \gets schedule(stamTasks)$
\STATE $seed \gets nextRand()$
\STATE $violations \gets 0$
\STATE $stamViolations \gets 0$
\STATE $setRandomSeed(seed)$
\FOR{$j = 1$ \TO $n$}
\STATE $v \gets simulate(taskList, schedule)$
\STATE $violations \gets violations + v$
\ENDFOR
\STATE $setRandomSeed(seed)$
\FOR{$j = 1$ \TO $n$}
\STATE $v \gets simulate(taskList, stamSchedule)$
\STATE $stamViolations \gets stamViolations + v$
\ENDFOR
\ENDFOR
\end{algorithmic}
\caption{Simulation Execution Framework}
\end{algorithm}

The tasks are generated with random periods, durations, and energy requirements.  The periods and durations are distributed uniformly in discrete time steps measured in days, ranging respectively from 10 to 40 and from 1 to 4.  The energy is half-normally distributed, and proportional to the task's period (\emph{i.e.} a task requiring high energy is expected to run at a low frequency).

A random task list and its corresponding virtual task list generated by \textsc{STAM} are generated reiteratively until both lists are temporally schedulable.  We consider a task list temporally schedulable when its CPU utilization $U$ is less than 100\%.  The utilization is calculated with equation~\ref{eqn:utilization}, where $k$ is the number of tasks, $d_i$ is the duration of the $i^{th}$ task, and $p_i$ is the period of the $i^{th}$ task. [source]
\begin{equation}
\label{eqn:utilization}
U = \sum_{i=1}^{k} \frac{d_i}{p_i}
\end{equation}
Talk about energy model.

Talk about storage model.

Talk about simulation results