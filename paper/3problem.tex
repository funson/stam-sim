\section{ Problem Formulation} \label{sec:problem}

To model task execution in the sensor nodes, we assume the following: 
\begin{itemize}
	\item (A1) The requests for all tasks are periodic, with constant interval between requests. Such tasks are also called recurring tasks. 
	\item (A2) Each request of a task has a hard deadline, which is defined as the time when the next request for the task arrives. 
	\item (A3) A task has constant run-time. Run-time refers to the time to execute the task without interruption. We assume that the priority of tasks may change, but once being executed, a task cannot be interrupted. 
	\item (A4) The task drains energy with a constant rate during its execution time\footnote{Energy consumption on sensor nodes largely depends on the operations of peripheral devices (e.g., sensors and wireless transmitters) associated with the task rather than executing code in the microprocessor.}.       
	\item (A5) The tasks are independent in that requests for a given task do not depend on the initialization or the completion of requests for other tasks.
	\item (A6) The sensor node includes energy harvester to supply power. It also has an energy storage module (capacitors or rechargeable batteries) with the maximum capacity of $C$. 
	%\item (A6) A schedule exists that satisfies the timing requirements of all tasks.
\end{itemize}

%To model the energy harvester component we assume the following:
 %\begin{itemize}
%	\item (A7) The energy harvester provides energy source to the node with a power function $P_S(t)$, which can be modeled as a stationary random process during the time in consideration.
%	\item (A8) The sensor node also includes an energy storage module (capacitors or rechargeable batteries) with the maximum capacity of $C$. 
%\end{itemize}
 
We can denote a set of recurring tasks by $\{\tau_1, \tau_2, \ldots, \tau_n\}$, with each task represented by a tuple $\tau_i = <T_i, D_i, E_i>$, where $T_i$ denotes the periodic interval time between requests for the task, $D_i$ denotes the task's duration, and $E_i$ denotes the task's energy consumption per time unit. For a set of tasks scheduled according to some scheduling algorithm, we say that a \textit{task violation} occurs at time $t$ if the node's energy level drops to zero or $t$ is the deadline of an unfulfilled request.  


%We denote the harvester component by $<C(t), P_S(t)>$, where $C(t)$ is the energy capacity at time $t$ and $P_S(t)$ is the stochastic energy input function. For a set of tasks scheduled according to some scheduling algorithm, we say that a \textit{violation} occurs at time $t$ if the node's energy level drops to zero.

In this paper we consider the question of how best to schedule tasks to reduce the likelihood of task violations.