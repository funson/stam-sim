\documentclass[10pt,conference]{IEEEtran}
%\documentclass[conference]{IEEEtran}

\makeatletter
\def\ps@headings{%
\def\@oddhead{\mbox{}\scriptsize\rightmark \hfil \thepage}%
\def\@evenhead{\scriptsize\thepage \hfil \leftmark\mbox{}}%
\def\@oddfoot{}%
\def\@evenfoot{}}
\makeatother

\pagestyle{headings}

\newcommand{\nop}[1]{}
\usepackage{graphicx}
\usepackage{epsfig}
\usepackage{balance}
\usepackage{algorithm}
\usepackage{algorithmic}
\usepackage[centertags]{amsmath}

\def\linespaces{0.95}
\def\baselinestretch{\linespaces}


\begin{document}

\title{Scheduling Recurring Tasks in Energy Harvesting Sensor Nodes}
\author{David Audet\\david's uvic email \and Leandro Collares\\leco@uvic.ca \and Neil MacMillan\\nrqm@uvic.ca \and Kui Wu\\wkui@ieee.org}

\maketitle


\begin{abstract}
In this paper we consider the problem of periodic task scheduling in the presence of energy restraints.  In particular, we focus on systems with stochastic energy sources such as photovoltaic cells, and we present two energy-aware scheduling algorithms that reduce the likelihood of a system running out of energy.  Our algorithms, called \emph{Smooth to Average Method} (STAM) and \emph{Smooth to Full Utilization} (STFU) are static schedulers that do not require prescience of the incoming energy to operate effectively.
\end{abstract}

\begin{IEEEkeywords} Real-Time Scheduling, Recurring Tasks, Energy Harvester, Sensors
\end{IEEEkeywords}


\section{Introduction}\label{sec:introduction}

A wireless sensor network (\textsc{wsn}) consists of collaborating embedded devices (sensor nodes) with capabilities of sensing, computation and communication
\cite{sudevalyam2010energy}. Wireless sensor networks can be deployed for a plethora of purposes such as habitat \cite{mainwaring2002wireless},
earthquake \cite{suzuki2007earthquake} and health \cite{saadaoui2007architecture} monitoring.

Most sensor nodes do not depend on existing infrastructure and are powered by batteries, which makes initial deployment in remote locations
convenient. On the other hand, the lifetime of these embedded devices is limited by the amount of energy that can be stored in the batteries. Furthermore, 
the number of sensors and their locations might render the activity of replacing nodes' batteries unfeasible \cite{moser2007real}. 

Intensive research has been conducted on energy harvesting as a way to extend the lifetime of wireless sensor networks. Several types of energy such as
solar, eolic (wind), vibrational, and thermal among others can be scavenged from the surroundings of a sensor node to replenish its battery \cite{roundy2004power}. 
Promising as it may seem, energy harvesting poses new challenges to the scientific community \cite{lu2010accurate}:

\begin{itemize}
	\item Environmental energy sources behave stochastically, which means it is not possible to know beforehand the exact amount of available energy.
	\item Conventional task scheduling techniques were not designed for energy-limited scenarios and cannot deal properly with the uncertainty in energy availability.
\end{itemize}

This paper evaluates the performance of two task scheduling techniques, Earliest Deadline First (\textsc{edf}) and the Lazy Scheduling
Algorithm (\textsc{lsa}), for a single node under energy-limited conditions. The two techniques were conceived for unlimited and limited energy scenarios respectively \cite{moser2007real}. We also present two new scheduling techniques, the Smooth to Averate Method
(\textsc{stam}) and Smooth to Full Utilization (\textsc{stfu}), designed to handle bursts in power consumption and address both time and energy constraints adequately without requiring a precise model of incoming energy.

We consider solar energy scavenging through photo-voltaic conversion, as it provides the highest power density of conventional environmental energy harvesting techniques
\cite{raghunathan2005design}. We use a Markov model based on a matrix of transition probabilities for three radiation intensity states to predict
the power provided by the harvesting unit \cite{poggi2000stochastic}.

\section{Related Work} \label{sec:related work}
In \oldstylenums{1973}, Liu and Layland~\cite{Liu73} presented two classical scheduling algorithms, rate-monotonic priority assignment and deadline-driven scheduling, 
and assess their performance based on processor utilization in unlimited energy scenarios. Moser \emph{et. al.}~\cite{moser2007real} described energy-aware \textsc{lsa} scheduling
and proved that it optimally deals with time and energy constraints in a system whose energy storage is replenished predictably. 

Power management algorithms based on duty-cycling between active and low-power modes of sensor nodes with energy harvesting capabilities are 
presented by Kansal \emph{et. al.}~\cite{kansal2007power}. Niyato \emph{et. al.}~\cite{niyato2007sleep} investigate the impact of different sleep and wake-up strategies on data communication
among solar-powered nodes. In~\cite{vigorito2007adaptive} Vigorito \emph{et. al.} propose an adaptive duty-cycling algorithm that ensures that power supply to sensor
nodes is kept within operational levels regardless of changing environmental conditions. In~\cite{moser2007adaptive} Moser \emph{et. al.} also presents an adaptive power management
model that can be customized to address different constraints and optimization objectives in energy harvesting systems.

Predicting stochastic energy sources is problematic.  Lu \emph{et. al.}~\cite{lu2010accurate} assess three prediction techniques for real-time systems: regression analysis,
moving average and exponential smoothing. Recas~\emph{et. al.}~\cite{recas2000hollows} employ the Weather-Conditioned Moving Average (\textsc{wcma}) model, which adapts to seasonal changes in
solar power harvesting as well as sudden weather changes. Moser \emph{et. al.}~\cite{moser2007real} introduced energy variability curves to predict accurately
the power provided by a harvesting unit. In~\cite{susu2008stochastic} Susu \emph{et. al.} use a discrete-time Markov chain in which only transitions between 
consecutive states, representing energy levels, are allowed. On the other hand, Niyato \emph{et. al.}~\cite{niyato2007sleep} make use of a Markov chain model 
that takes into consideration the influence of clouds and wind on solar radiation intensity. 

The majority of energy harvesting systems are based on rechargeable batteries, as noted by Sudevalayam and Kulkarni~\cite{sudevalyam2010energy}.
However, Simjee and Chou~\cite{simjee2007everlast} add that there has been increasing interest in supercapacitors, which theoretically have infinite charge cycles and high power densities. In~\cite{jiang2005perpetual}, Jiang \emph{et. al.} present a hybrid implementation based on two supercapacitors and one rechargeable battery.

\section{ Problem Formulation} \label{sec:problem}

To model task execution in the sensor nodes, we assume the following: 
\begin{itemize}
	\item (A1) The requests for all tasks are periodic, with constant interval between requests. Such tasks are also called recurring tasks. 
	\item (A2) Each request of a task has a hard deadline, which is defined as the time when the next request for the task arrives. 
	\item (A3) A task has constant run-time. Run-time refers to the time to execute the task without interruption. We assume that the priority of tasks may change, but once being executed, a task cannot be interrupted. 
	\item (A4) The task drains energy with a constant rate during its execution time\footnote{Energy consumption on sensor nodes largely depends on the operations of peripheral devices (e.g., sensors and wireless transmitters) associated with the task rather than executing code in the microprocessor.}.       
	\item (A5) The tasks are independent in that requests for a given task do not depend on the initialization or the completion of requests for other tasks.
	\item (A6) The sensor node includes an energy harvester to supply power. It also has an energy storage module (capacitors or rechargeable batteries) with the maximum capacity of $C$. 
	%\item (A6) A schedule exists that satisfies the timing requirements of all tasks.
\end{itemize}

%To model the energy harvester component we assume the following:
 %\begin{itemize}
%	\item (A7) The energy harvester provides energy source to the node with a power function $P_S(t)$, which can be modeled as a stationary random process during the time in consideration.
%	\item (A8) The sensor node also includes an energy storage module (capacitors or rechargeable batteries) with the maximum capacity of $C$. 
%\end{itemize}
 
We can denote a set of recurring tasks by $\{\tau_1, \tau_2, \ldots, \tau_n\}$, with each task represented by a tuple $\tau_i = <T_i, D_i, P_i>$, where $T_i$ denotes the periodic interval time between requests for the task, $D_i$ denotes the task's execution duration, and $P_i$ denotes the task's energy consumption per time unit (\emph{i.e.} power consumption). For a set of tasks scheduled according to some scheduling algorithm, we say that a \textit{task violation} occurs at time $t$ if the node's energy level drops to zero or $t$ is the deadline of an unfulfilled request.  


%We denote the harvester component by $<C(t), P_S(t)>$, where $C(t)$ is the energy capacity at time $t$ and $P_S(t)$ is the stochastic energy input function. For a set of tasks scheduled according to some scheduling algorithm, we say that a \textit{violation} occurs at time $t$ if the node's energy level drops to zero.

In this paper we consider the question of scheduling tasks to reduce the likelihood of task violations.



\section{Algorithm} \label{sec:algorithm}

As a means of producing fewer energy violations when scheduling system tasks, We have developed a technique for reducing energy violations when scheduling tasks, called the \emph{Smooth to Average Method} (\textsc{STAM}). This heuristic takes into account the average energy requirements per unit time of all tasks in a given task list. We generate a set of equivalent virtual tasks by increasing the duration of any task that uses greater than average energy per unit time until its energy used per unit time is near the average. In these virtual tasks, the total energy remains the same as for real tasks' energy but it is spread over a longer duration.  Virtual tasks cannot be scheduled to run at the same time.  Once the virtual tasks are scheduled, the real tasks are inserted at the end of the corresponding virtual task's timeslot.  Thus a real task that consumes high energy is guaranteed to run after an idle period, reducing the likelihood that the system will run out of energy when the task runs.

\begin{algorithm}[htb]
\label{stamalg}
\begin{algorithmic}
\STATE INPUT: $realTasks$ \COMMENT {list of [period, duration, energy]} 
\STATE INPUT: $N$ \COMMENT {number of tasks}
\STATE OUTPUT: $vTasks$ \COMMENT {same format as $realTasks$}
\STATE $meanEnergy \gets mean(realTasks[:,3])$
\FOR{$i = 1$ \TO $N$}
\IF{$taskList[i,3] > meanEnergy$}
\STATE $taskEnergy \gets realTasks[i, 2] * realTasks[i,3]$
\STATE $vDuration \gets \lceil \frac{taskEnergy}{meanEnergy} \rceil$
\STATE $vEnergy \gets \frac{taskEnergy}{vDuration}$
\STATE $vTasks[i,:] \gets [taskList[i,1]~~vDuration~~vEnergy]$
\ELSE
\STATE $vTasks[i,:] \gets taskList[i,:]$
\ENDIF
\ENDFOR
\end{algorithmic}
\caption{Generate \textsc{STAM} Task List}
\end{algorithm}

The \textsc{STAM} algorithm calculates the energy consumption of each task by multiplying its runtime by the task's energy consumption per time unit. After taking the mean energy consumption across all of the tasks in the task list, each task is compared to the this value and virtual tasks are generated accordingly. If the given task's energy consumption is above the mean energy value, the virtual duration is calculated by taking the ceiling of the energy area of the task divided by the calculated energy mean. This will extend the duration of the virtual task allowing the total energy consumed to be more evenly distributed across the duration of the task's runtime. If the given task's energy consumption is below the calculated energy mean, the algorithm is unable to perform any smoothing and  will use the unchanged physical task to generate a schedule.
\begin{figure}[htb]
\includegraphics[scale=0.38]{stamtasks.png}
\caption{EDF schedules for four tasks and their STAM equivalents}
\label{fig:stamtaskplot}
\end{figure}

Figure~\ref{fig:stamtaskplot} shows 

\section{Simulation Evaluation} \label{sec:simulation}

We have developed a simulation framework for comparing the \textsc{STAM} task scheduling to traditional scheduling algorithms.  Our simulation includes a stochastic energy harvesting process, a random task list and \textsc{STAM} task list generator, the scheduling processes, and an execution process.  We execute $n$ simulations on one task list per run, and generate task lists for $r$ runs.  Each task list consists of $k$ tasks.  The framework is as follows:

\begin{algorithm}[h]
\begin{algorithmic}
\FOR{$i = 1$ \TO $r$}
\STATE $setRandomSeed(i + offset)$
\STATE $taskList, stamTasks \gets generateTasks()$
\STATE $schedule \gets schedule(taskList)$
\STATE $stamSchedule \gets schedule(stamTasks)$
\STATE $seed \gets nextRand()$
\STATE $violations \gets 0$
\STATE $stamViolations \gets 0$
\STATE $setRandomSeed(seed)$
\FOR{$j = 1$ \TO $n$}
\STATE $v \gets simulate(taskList, schedule)$
\STATE $violations \gets violations + v$
\ENDFOR
\STATE $setRandomSeed(seed)$
\FOR{$j = 1$ \TO $n$}
\STATE $v \gets simulate(taskList, stamSchedule)$
\STATE $stamViolations \gets stamViolations + v$
\ENDFOR
\ENDFOR
\end{algorithmic}
\caption{Simulation Execution Framework}
\end{algorithm}

The tasks are generated with random periods, durations, and energy requirements.  The periods and durations are distributed uniformly in discrete time steps measured in days, ranging respectively from 10 to 40 and from 1 to 4.  The energy is half-normally distributed, and proportional to the task's period (\emph{i.e.} a task requiring high energy is expected to run at a low frequency).

A random task list and its corresponding virtual task list generated by \textsc{STAM} are generated reiteratively until both lists are temporally schedulable.  We consider a task list temporally schedulable when its CPU utilization $U$ is less than 100\%.  The utilization is calculated with equation~\ref{eqn:utilization}, where $k$ is the number of tasks, $d_i$ is the duration of the $i^{th}$ task, and $p_i$ is the period of the $i^{th}$ task. [source]
\begin{equation}
\label{eqn:utilization}
U = \sum_{i=1}^{k} \frac{d_i}{p_i}
\end{equation}
Talk about energy model.

Talk about storage model.

Talk about simulation results

\section{Conclusion} \label{sec:conclusion}

We have presented novel algorithms appropriate for the scheduling of hard real time periodic tasks 
for sensing devices powered through energy harvesting.  
Unlike most previous work in this area, our approach to task scheduling is static, 
and does not require a model of energy replenishment. 
%Therefore the algorithms we have presented here are of a complexity suitable for implementation on a %typical, resource poor sensing device.  

Experiments conducted through simulations that incorporate a \emph{dynamic} energy replenishment model 
show that our scheduling algorithms perform better than classic, 
non-energy-aware, static scheduling algorithms. 
Furthermore, our static scheduling approaches perform at a level similar to the current state of the art, 
energy-aware scheduling algorithms that require prediction models such as proposed by Moser \emph{et al.} \cite{moser2007real}.  

As an on-going project, we have implemented a solar-powered wireless sensor node, and we are developing current monitoring to accurately measure energy charging and discharging rate. In future work, we will evaluate our approach by running a data acquisition application on our solar-powered sensor nodes. 

\section*{Acknowledgements}

We acknowledge the Natural Sciences and Engineering Research Council of Canada for their funding. 






%\balance




\bibliographystyle{abbrv}

\bibliographystyle{plain}
%\small \baselineskip 9pt
\bibliography{referenceEnergy}

\end{document}
