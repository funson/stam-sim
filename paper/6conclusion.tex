\section{Conclusion} \label{sec:conclusion}

We have presented novel algorithms appropriate for the scheduling of hard real time periodic tasks 
for sensing devices powered through energy harvesting.  
Unlike most previous work in this area, our approach to task scheduling is static, 
and does not require a model of energy replenishment. 
%Therefore the algorithms we have presented here are of a complexity suitable for implementation on a %typical, resource poor sensing device.  

Experiments conducted through simulations that incorporate a \emph{dynamic} energy replenishment model 
show that our scheduling algorithms perform better than classic, 
non-energy-aware, static scheduling algorithms. 
Furthermore, our static scheduling approaches perform at a level similar to the current state of the art, 
energy-aware scheduling algorithms that require prediction models such as proposed by Moser \emph{et al.} \cite{moser2007real}.  

As an on-going project, we have implemented a solar-powered wireless sensor node, and we are developing current monitoring to accurately measure energy charging and discharging rate. In future work, we will evaluate our approach by running a data acquisition application on our solar-powered sensor nodes. 

\section*{Acknowledgements}

We acknowledge the Natural Sciences and Engineering Research Council of Canada for their funding. 

