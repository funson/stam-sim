
\section{Algorithm} \label{sec:algorithm}

As a means of producing fewer energy violations when scheduling system tasks, we used a technique that we have named the \emph{Smooth to Average Method (STAM)}. This scheduling heuristic takes into account the average overall energy requirements of all tasks in a given task set. This energy average is then used to modify the given taskset to ensure that the energy consumption of each task is at or near the calculated threshold. This smoothing of energy consumption is done by creating a virtual task set with modified start and end times to allow energy consumption to be distributed over a larger time interval than the original physical task runtime. To accomplish this energy smoothing, tasks are still executed at their physical runtimes, but during the virtual time period, energy is able to still be generated by the environmental source. This reduces the net energy consumption over the runtime of each task.

\begin{algorithm}[tb]
\label{stamalg}
\begin{algorithmic}
\STATE INPUT: $realTasks$ \COMMENT {list of [period, duration, energy]} 
\STATE INPUT: $N$ \COMMENT {number of tasks}
\STATE OUTPUT: $vTasks$ \COMMENT {same format as $realTasks$}
\STATE $meanEnergy \gets mean(realTasks[:,3])$
\FOR{$i = 1$ \TO $N$}
\IF{$taskList[i,3] > meanEnergy$}
\STATE $taskEnergy \gets realTasks[i, 2] * realTasks[i,3]$
\STATE $vDuration \gets \lceil \frac{taskEnergy}{meanEnergy} \rceil$
\STATE $vEnergy \gets \frac{taskEnergy}{vDuration}$
\STATE $vTasks[i,:] \gets [taskList[i,1]~~vDuration~~vEnergy]$
\ELSE
\STATE $vTasks[i,:] \gets taskList[i,:]$
\ENDIF
\ENDFOR
\end{algorithmic}
\caption{Smooth to Average Method (STAM)}
\end{algorithm}

The actual STAM algorithm that we implemented in Matlab calculates the energy consumption of each task by multiplying its runtime by the duration of one execution of the task. After taking the mean energy consumption across all of the tasks in the task set, each task is compared to the this value and virtual tasks are scheduled accordingly for each task. If the given task's energy consumption is above the mean energy value, the virtual duration will be calculated by taking the ceiling of the energy area of the task divided by the calculated energy mean. This will extend the duration of the virtual task allowing the total energy consumed to be more evenly distributed across the duration of the task's runtime. If the given task's energy consumption is below the calculated energy mean, we are unable to perform any smoothing and  will use the unchanged physical task for the simulation procedure.