\section{Introduction}\label{sec:introduction}

A wireless sensor network (\textsc{wsn}) consists of collaborating sensor nodes with capabilities of sensing, computation and communication~\cite{sudevalyam2010energy}. Wireless sensor networks can be deployed for a plethora of purposes such as habitat monitoring~\cite{mainwaring2002wireless}, earthquake detection~\cite{suzuki2007earthquake}, or healthcare~\cite{saadaoui2007architecture}. 

For ease of deployment, wireless sensor networks usually do not rely on existing infrastructure, and sensor nodes are typically battery powered. Therefore, the lifetime of these embedded devices is limited by the amount of energy that can be stored in the batteries. Furthermore, in many applications such as forest monitoring, the number of sensors and their locations might render the activity of replacing nodes' batteries infeasible or very costly~\cite{moser2007real}. 
There is a need for green solutions capable of powering sensor network applications 
with ambient energy. 

To solve the above problem, intensive research has been conducted on energy harvesting as a way to extend the lifetime of wireless sensor networks. Several types of energy such as solar, eolic (wind), vibrational, and thermal among others can be scavenged from the surroundings of a sensor node to replenish its battery~\cite{roundy2004power}. Promising as it may seem, energy harvesting poses new challenges to the scientific community~\cite{lu2010accurate}:

\begin{itemize}
	\item Environmental energy sources behave stochastically, making the accurate prediction of incoming energy levels very difficult.
	\item Conventional task scheduling techniques were not designed for energy-limited scenarios and cannot deal properly with uncertainty in energy availability.
\end{itemize}

It has been pointed out that the traditional scheduling methods, such as Earliest Deadline First (\textsc{edf}), may not work well under energy-limited conditions~\cite{moser2007real}, and as such new algorithms such as the Lazy Scheduling Algorithm (\textsc{lsa}) have been proposed to ``solve" the problem~\cite{moser2007real}. 
Although it has been theoretically proven that LSA is optimal, it requires an accurate prediction on the incoming energy source to operate well. 
Energy prediction, however, is non-trivial, and it is challenging 
to implement a suitably intelligent  prediction algorithm 
on a typical sensor node due to the computational resources available on such a platform. 
Installing a pre-trained energy prediction model does not work either, 
because such a model depends on where and when the model was built and may 
not generalize when the sensors are deployed in different places and 
function over a long time period. 

In this paper, we make contributions by proposing two new scheduling techniques, the Smooth to Average Method (\textsc{stam}) and Smooth to Full Utilization (\textsc{stfu}), to handle energy uncertainty and deadline constraints without relying on any energy prediction model\footnote{Later in this paper we build an energy charging model for solar energy harvesting; however this model is purely for the purpose of performance comparison and in a real implementation such a model is not required.}.
In our simulation-based evaluations, we consider solar energy scavenging through photo-voltaic conversion, as it provides the highest power density of conventional environmental energy harvesting techniques~\cite{raghunathan2005design}. 

In the remainder of the paper we first discuss related work on energy-aware scheduling, 
and formalize the problem of periodic real-time task scheduling. 
We then present two static, energy-aware scheduling algorithms and give
an evaluation of their performance compared to related work through simulations. 
Finally, we give some concluding remarks. 

%We use a Markov model based on a matrix of transition probabilities for three radiation %intensity states to predict the power provided by the harvesting unit \cite
%{poggi2000stochastic}.
