\section{Related Work} \label{sec:related work}
In \oldstylenums{1973}, Liu and Layland~\cite{Liu73} presented two classical scheduling algorithms, rate-monotonic priority assignment and deadline-driven scheduling, 
and assess their performance based on processor utilization in unlimited energy scenarios. Moser \emph{et. al.}~\cite{moser2007real} described energy-aware \textsc{lsa} scheduling
and proved that it optimally deals with time and energy constraints in a system whose energy storage is replenished predictably. 

Power management algorithms based on duty-cycling between active and low-power modes of sensor nodes with energy harvesting capabilities are 
presented by Kansal \emph{et. al.}~\cite{kansal2007power}. Niyato \emph{et. al.}~\cite{niyato2007sleep} investigate the impact of different sleep and wake-up strategies on data communication
among solar-powered nodes. In~\cite{vigorito2007adaptive} Vigorito \emph{et. al.} propose an adaptive duty-cycling algorithm that ensures that power supply to sensor
nodes is kept within operational levels regardless of changing environmental conditions. In~\cite{moser2007adaptive} Moser \emph{et. al.} also presents an adaptive power management
model that can be customized to address different constraints and optimization objectives in energy harvesting systems.

Predicting stochastic energy sources is problematic.  Lu \emph{et. al.}~\cite{lu2010accurate} assess three prediction techniques for real-time systems: regression analysis,
moving average and exponential smoothing. Recas~\emph{et. al.}~\cite{recas2000hollows} employ the Weather-Conditioned Moving Average (\textsc{wcma}) model, which adapts to seasonal changes in
solar power harvesting as well as sudden weather changes. Moser \emph{et. al.}~\cite{moser2007real} introduced energy variability curves to predict accurately
the power provided by a harvesting unit. In~\cite{susu2008stochastic} Susu \emph{et. al.} use a discrete-time Markov chain in which only transitions between 
consecutive states, representing energy levels, are allowed. On the other hand, Niyato \emph{et. al.}~\cite{niyato2007sleep} make use of a Markov chain model 
that takes into consideration the influence of clouds and wind on solar radiation intensity. 

The majority of energy harvesting systems are based on rechargeable batteries, as noted by Sudevalayam and Kulkarni~\cite{sudevalyam2010energy}.
However, Simjee and Chou~\cite{simjee2007everlast} add that there has been increasing interest in supercapacitors, which theoretically have infinite charge cycles and high power densities. In~\cite{jiang2005perpetual}, Jiang \emph{et. al.} present a hybrid implementation based on two supercapacitors and one rechargeable battery.