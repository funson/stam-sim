%\documentclass[10pt,conference]{IEEEtran}
\documentclass[conference]{IEEEtran}

\makeatletter
\def\ps@headings{%
\def\@oddhead{\mbox{}\scriptsize\rightmark \hfil \thepage}%
\def\@evenhead{\scriptsize\thepage \hfil \leftmark\mbox{}}%
\def\@oddfoot{}%
\def\@evenfoot{}}
\makeatother

\pagestyle{headings}

\newcommand{\nop}[1]{}
\usepackage{stmaryrd}
\usepackage{bbding}
% \usepackage{algorithmic}
% \usepackage{algorithm}
\usepackage{graphicx}
\usepackage{epsfig}
\usepackage{subfigure}
\usepackage{multirow}
\usepackage{balance}
%\usepackage{bibentry}
%\usepackage[square,comma]{natbib}

\usepackage[centertags]{amsmath}

\newtheorem{definition}{Definition}
\newtheorem{remark}{Remark}
\newtheorem{Rule}{Rule}
\newtheorem{proposition}{Proposition}
\newtheorem{lemma}{Lemma}
\newtheorem{theorem}{Theorem}
\newtheorem{Theorem}{Theorem}
\newtheorem{corollary}{Corollary}
\newtheorem{result}{Result}
\newtheorem{question}{Question}
\newtheorem{proof}{Proof}
\newtheorem{example}{Example}

\def\linespaces{0.95}
\def\baselinestretch{\linespaces}

\nop{

\documentclass[11pt,conference]{IEEEtran}

\usepackage{graphicx}
\usepackage{balance}
\usepackage{amsmath}
\begin{document}
\newcommand{\qed}{\mbox{\rule[0pt]{1.0ex}{1.0ex}}}
\def\boxend{\hspace*{\fill} $\QED$}

\newtheorem{definition}{Definition}
\newtheorem{example}{Example}
\newtheorem{theorem}{Theorem}
\newtheorem{problem}{Problem}
\newtheorem{lemma}{Lemma}
\newtheorem{proposition}{Proposition}
\newtheorem{remark}{Remark}

% \newtheorem{assumption}{Assumption}
% \newtheorem{property}{Property}
% \newtheorem{conjecture}{Conjecture}
}


\newcounter{line}
\input epsf
% \renewcommand{\baselinestretch}{0.95}

\hyphenation{op-tical net-works semi-conduc-tor}

\begin{document}

\title{Scheduling for Recurring Tasks in Energy Harvesting Sensor Nodes}

\nop{
\author{\IEEEauthorblockN{A \IEEEauthorrefmark{1},
B \IEEEauthorrefmark{2}, and C \IEEEauthorrefmark{3}}\\
\IEEEauthorblockA{\IEEEauthorrefmark{1}Dept. of Computer Science, University of Victoria, B.C., Canada}
}
}
%\date{}
%\CopyrightYear{2008}

\maketitle

\begin{abstract}

\end{abstract}

% A category with the (minimum) three required fields
%\category{C.4}{Performance of Systems}{Modeling techniques}
%\category{H.1}{Models and Principles}{Miscellaneous}
%A category including the fourth, optional field follows...
%\category{D.2.8}{Software Engineering}{Metrics}[complexity measures, performance measures]

%\terms{Theory, Performance}

\begin{IEEEkeywords} Real-Time Scheduling, Recurring Tasks, Energy Harvester, Sensors
\end{IEEEkeywords}

\section{Introduction}\label{sec:introduction}

\nop{
The rest of the paper is organized as follows. We introduce the related work in Section~\ref{sec:work} and the background knowledge in Section~\ref{sec:background}.
 In Section~\ref{sec:Tradeoff}, we discuss the tradeoffs in model transform in stochastic network calculus. In Section~\ref{sec:impact},
 we investigate the impact of model transform on performance evaluation. In Section~\ref{sec:integration}, we present a simple methodology for model 
buildup and performance analysis. In Section~\ref{sec:example}, we use an example to illustrate some ``tricks" in the application of stochastic network 
calculus. The paper is concluded in Section~\ref{sec:conclusion}.  
}

A wireless sensor network (\textsc{wsn}) consists of collaborating embedded devices (sensor nodes) with capabilities of sensing, computation and communication
\cite{sudevalyam2010energy}. Wireless sensor networks can deployed for a plethora of purposes such as habitat \cite{mainwaring2002wireless},
earthquake \cite{suzuki2007earthquake} and health \cite{saadaoui2007architecture} monitoring.

Most of sensor nodes do not depend on existing infrastructure and are powered by batteries, which makes initial deployment in remote locations somewhat
convenient. On the other hand, the lifetime of these embedded devices is limited by the amount of energy that can be stored in batteries. Furthermore, 
the huge amount of sensors and their locations might render the activity of replacing nodes' batteries unfeasible \cite{moser2007real}. 

Intensive research has been conducted on energy harvesting as a way of extending the lifetime of wireless sensor networks. Several types of energy such as
solar, eolic, vibrational, thermal among others can be scavenged from the surroundings of a sensor node to replenish its battery \cite{roundy2004power}. 
Promising as it may seem, energy harvesting poses new challenges to the scientific community \cite{lu2010accurate}:

\begin{itemize}
	\item Environmental energy sources present stochastic behaviour, which means it is not possible to know beforehand the exact amount of available energy.
	\item Conventional task scheduling techniques were not designed for energy-limited scenarios and cannot deal with the uncertainty in energy availability suitably.
\end{itemize}

This paper focuses on only one sensor node and evaluates the performance of two task scheduling techniques, Earliest Deadline First (\textsc{edf}) and Lazy Scheduling
Algorithm (\textsc{lsa}) under energy-limited conditions, namely battery capacity and environmental energy harvesting. The aforementioned techniques, \textsc{edf} 
and \textsc{lsa}, were conceived for unlimited and limited energy scenarios respectively \cite{moser2007real}. We also present a new scheduling technique, xx xx xx xx  
(\textsc{stam}), designed to avoid power consumption bursts (??) and address both time and energy constraints adequately.  

Even though several energy forms could have been used, solar energy scavenging through photo-voltaic conversion is employed as it provides the highest power density 
\cite {raghunathan2005design}. A Markov model based on a matrix of transition probabilities for three radiation intensity states is utilized to predict
the power provided by the harvesting unit \cite{poggi2000stochastic}.

The rest of the paper is organized as follows...

Intro
Related Work
Problem Formulation
Algorithm
Simulation Evaluation
Conclusion







 



 

\section{Related Work}
\label{sec:work}

% cite{Liu73, Moser07, gorlatova2010networking, sudevalyam2010energy}.

\section{Model and Problem Formulation} \label{sec:model}

We make the following assumptions on tasks execution in the sensor nodes, which are generally true but might not be absolutely necessary for certain application scenarios. 
\begin{itemize}
	\item (A1) The requests for all tasks are periodic, with constant interval between requests. 
	\item (A2) Each request of a task has a hard deadline, which is defined as the time when the next request for the task arrives. 
	\item (A3) A task has constant run-time and consumes constant energy\footnote{Energy consumption on sensor nodes largely depends on the operations of peripheral devices (e.g., sensors, wireless transmitters) associated with the task instead of executing code in the microprocessor.}. Run-time refers to the time which is taken by the microprocessor to execute the task without interruption.
	\item (A4) The task drains the power with a constant rate during its execution time.       
	\item (A5) The tasks are independent in that requests for a given task do not depend on the initialization or the completion of requests for other tasks.	
\end{itemize}

The sensor nodes consist of an energy harvestor component with the following assumptions
 \begin{itemize}
	\item (A6) The energy harvestor provides energy source to the node with a power function $P_S(t)$, which can be modeled as a stationary random process during the time in consideration.
	\item (A7) The sensor node also includes an energy storage (capacitors or rechargeable batteries) with the maximum capacity of $C$. 
\end{itemize}
 
We can denote a set of recurring tasks by $\{\tau_1, \tau_2, \ldots, \tau_n\}$, with each task represented by a tuple $\tau_i = <I_i, E_i, T_i>$, where $I_i$ denotes the periodic interval time between requests, $E_i$ denotes the energy consumption of the task, and $T_i$ denotes the run-time of the task. We can denote the energy component by $<C, P_S(t)>$. For a set of tasks scheduled according to some scheduling algorithm, we say that an \textit{overflow} occurs at time $t$ if $t$ is the deadline of an unfulfilled request or $t$ is the time when the energy level of the node drops to zero. A set of recurring tasks is scheduable if they can be scheduled so that no overflow ever occurs. 

We are interested in answering the following question: Is a given set of recurring tasks schedulable with a high probability?  

%\balance

\bibliographystyle{abbrv}

\bibliographystyle{plain}
%\small \baselineskip 9pt
\bibliography{referenceEnergy}

\end{document}
