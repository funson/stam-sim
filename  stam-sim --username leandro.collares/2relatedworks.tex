\section{Related Work} \label{sec:related work}

Important early work in real time scheduling by Liu and Layland~\cite{Liu73} presented two classic scheduling algorithms,
rate-monotonic priority assignment and deadline-driven scheduling, and assessed their performance based on processor
utilization. Their work, however, did not consider energy constraints. 
Moser \emph{et al.}~\cite{moser2007real}, in more recent work, described energy-aware \textsc{lsa} scheduling
and proved that it optimally deals with time and energy constraints in a system whose energy storage is replenished
\textit{predictably}. The suitability of this approach under realistic energy harvesting conditions, however, is unclear. 

Research into energy-aware algorithms for sensor nodes is an active area. Kansal \emph{et al.}~\cite{kansal2007power}
presented power management algorithms based on duty-cycling between active and low-power modes of sensor nodes 
with energy harvesting capabilities to achieve perennial operation at a desired performance level. 
Niyato \emph{et al.}~\cite{niyato2007sleep} investigated the impact of sleep and wake-up strategies 
on data communication among solar-powered nodes. These strategies are dependent on battery charge, solar radiation
level and number of packets in the queue. In~\cite{vigorito2007adaptive}, Vigorito \emph{et al.} proposed an adaptive
duty-cycling algorithm that ensures that power supplied to sensor nodes is kept within operational levels in several
energy harvesting scenarios. The
algorithm does not require previous information on the energy source dynamics and presents low computational
demands. In~\cite{moser2007adaptive}, Moser \emph{et al.} presented an adaptive power management
model that can be customized to address different constraints and optimization objectives in energy harvesting systems 
such as tradeoffs between communication and memory usage.


Predicting stochastic energy sources is non-trivial.  
Lu \emph{et al.}~\cite{lu2010accurate} assessed three prediction techniques, regression analysis,
moving average, and exponential smoothing, which meet the requirements imposed by a real-time energy
harvesting embedded system: high accuracy and low computation and memory demands. 
Recas~\emph{et al.}~\cite{recas2000hollows} employed the Weather-Conditioned Moving Average (\textsc{wcma}) model,
which adapts to seasonal changes in solar power harvesting as well as sudden weather changes. 
Moser \emph{et al.}~\cite{moser2007real} introduced energy variability curves to predict the power provided by a 
harvesting unit. Energy predictions based on these curves are highly accurate when sensor nodes utilization is low.
In~\cite{susu2008stochastic} Susu \emph{et al.} used a discrete-time Markov chain in which only transitions between 
consecutive states, representing energy levels generated by a solar panel, were allowed. This restriction hinges
on the fact that abrupt changes in the energy provided in a time step are highly improbable. 
On the other hand, Niyato \emph{et al.}~\cite{niyato2007sleep} made use of a Markov chain model that takes into
consideration the influence of clouds and wind on solar radiation intensity. 
Due to the limited computational and memory resources available on a typical sensor node, however,
implementing suitably accurate dynamic energy predication models appears challenging, making 
scheduling algorithms based on energy predication impractical. 

%The majority of energy harvesting systems are based on rechargeable batteries, as noted by Sudevalayam and Kulkarni~\cite{sudevalyam2010energy}.
%However, Simjee and Chou~\cite{simjee2007everlast} add that there has been increasing interest in %supercapacitors, which theoretically have infinite charge cycles and high power densities. 
In~\cite{jiang2005perpetual}, Jiang \emph{et. al.} present a hybrid implementation based on two supercapacitors and one rechargeable battery.

